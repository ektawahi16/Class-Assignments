\documentclass[a4paper, 12pt]{article}
\usepackage{graphicx}
\begin{document}
\title{Scribe}
\author{Ekta Wahi}
\date{November 1,2018}
\maketitle
\section{Bipartite Graphs}
\paragraph{}
 A bipartite graph (or bigraph) is a graph whose vertices can be divided into two disjoint and independent sets  U and V such that every edge connects a vertex in U to one in V. 
\paragraph{
G=(U,V,E) is used to denote a bipartite graph whose partition has the parts U and V, with E denoting the edges of the graph. }
\begin{center}
\includegraphics[width=0.5\textwidth]{bipartite.png}
\paragraph{}\captionof{ Figure 1.1 A Bipartite Graph} 
\end{center} 
\section{Bipartite Matching}
\paragraph{}A matching M is a subset of edges such that each node in V appears in at most one edge in M.
\paragraph{}
\textbf{Definition 2.1} (Maximum Matching) A maximum matching is a matching with the largest
possible number of edges; it is globally optimal.
\paragraph{}
\textbf{Definition 2.2} An alternating path with respect to M is a path that alternates between
edges in M and edges in E − M.
\paragraph{}
\textbf{Definition 2.3} An augmenting path with respect to M is an alternating path in which
the first and last vertices are exposed.}
\section{Problem}
\paragraph{}Given a Bipartite graph G=(U,V,E), Find a matching in G which has the maximum cardinality.
\\
\\
\paragpraph{\\{\textbf{BIPARTITE MATCHING(G)}}
\\M = \phi
\\{\textbf{repeat}}
\hspace{35pt}P =(Augmenting-Path(G, M))

\hspace{55pt}M = M \oplus P
\\{\textbf{until}} \hspace{40pt}P = \phi



\end{document}